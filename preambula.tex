       
              
%\usepackage[utf8]{inputenc}    % кодовая страница документа
%\usepackage[T2A]{fontenc}
\usepackage[russian]{babel}    % локализация и переносы
\usepackage{scrhack}           % remove warnings on float@addtolists
%\IfFileExists{pscyr.sty}{\usepackage{pscyr}}{}
%\usepackage{cmap}              % русский поиск в pdf
%\renewcommand{\rmdefault}{ftm} % Times New Roman
%\usepackage{amsfonts}          % шрифты AMS
%\usepackage{amsmath}       
%\usepackage{amssymb}
%\usepackage{amsthm}
%-synctex=1 -interaction=nonstopmode  --shell-escape %.tex
\usepackage{mathrsfs}
         
\usepackage[top=2.4cm, 
            bottom=2.8cm, 
            right=2.5cm,
            left=2.5cm,
            footskip=0.8cm]{geometry}
\usepackage{indentfirst}    % русский стиль: отступ первого абзаца раздела
%\usepackage{misccorr}      % точка в номерах заголовков
%\usepackage{graphicx}      % Работа с графикой \includegraphics{}
\usepackage{wrapfig}
%\usepackage{soul}          % Разряженный текст \so{} и подчеркивание \ul{}
%\usepackage{soulutf8}      % Поддержка UTF8 в soul
%\usepackage{fancyhdr}      % Для работы с колонтитулами
\usepackage{multirow}       % Аналог multicolumn для строк
\usepackage{rotating}       % Направление текста в таблице
\usepackage{ltxtable}       % Микс tabularx и longtable
\usepackage{tabulary}       % таблицы с фиксированной шириной
\usepackage{threeparttable} % таблицы со ссылками
\usepackage{paralist}       % Списки с отступом только в первой строчке
\usepackage[perpage]{footmisc} % Нумерация сносок на каждой странице с 1
\usepackage{hyperref}       % ссылки

%\setcounter{page}{3}
           
\usepackage{enumitem}

\setlist{topsep=2pt,
         partopsep=0pt,         
         parsep=0pt,
         itemsep=0pt,
         leftmargin=0.7cm}
\setlist[enumerate]{widest=0}
\usepackage{import}        % include с относительными путями

%\usepackage[cachedir=cache]{minted}        % подсветка синтаксиса
%\usemintedstyle{bw}                        % черно-белый стиль
 
%\usepackage[os=win]{menukeys}      % клавиатерные клавиши
%\changemenucolortheme{menus}{blacknwhite}
\pagestyle{plain}    % включить только номера страниц
% Getting title pages to the resultant pdfauthor
\usepackage{pdfpages}

%\setcounter{secnumdepth}{\sectionnumdepth} % выключаем номера подсекций
\addtocounter{tocdepth}{-1} % убираем подсекции из содержания
%\addtocounter{tocdepth}{1} % добавляем в содержание вплоть до subsubsections, DEBUG only!

\setlength{\parindent}{0.7cm}    % красная строка

% уменьшение расстояния до кода
%\newlength{\fancyvrbtopsep}
%\newlength{\fancyvrbpartopsep}
%\makeatletter
%\FV@AddToHook{\FV@ListParameterHook}{\topsep=\fancyvrbtopsep\partopsep=\fancyvrbpartopsep}
%\makeatother

%\setlength{\fancyvrbtopsep}{3pt}
%\setlength{\fancyvrbpartopsep}{3pt}

\makeatletter
\renewenvironment{quote}
               {\list{}{\listparindent=0pt   % whatever you need
                        \itemindent    \listparindent
                        \leftmargin=25pt     % whatever you need
                        \rightmargin=30pt    % whatever you need
                        \topsep=0pt      %%%%% whatever you need
                        \parsep        \z@ \@plus\p@}%
                \item\relax}
               {\endlist}
\makeatother

% line breaks in tables
\newcommand{\specialcell}[2][t]{%
  \begin{tabular}[#1]{@{}l@{}}#2\end{tabular}}

%\newtheoremstyle{examplestyle}      % <name>
%        {3pt}                       % <space above>
%        {3pt}                       % <space below>
%        {\small}                    % <body font>
%        {}                          % <indent amount}
%        {\bfseries}                 % <theorem head font>
%        {\bfseries.}                % <punctuation after theorem head>
%        {.5em}                      % <space after theorem head>
%        {}                          % <theorem head spec (can be left empty, meaning `normal')>
%\theoremstyle{examplestyle}
%\newtheorem{example}{Пример}[chapter]
%\theoremstyle{remark}
%\newtheorem{problem}{}[section]

% Chapter page numbers in TOC
\setkomafont{chapterentrypagenumber}{\nullfont}
\setkomafont{paragraph}{\normalfont\rmfamily\bfseries}

\renewcommand\addchaptertocentry[2]{%
  \IfArgIsEmpty{#1}
    {\addtocontents{toc}
        {\protect\begingroup
          \protect\setkomafont{chapterentrypagenumber}{\normalfont}%
          \protect\KOMAoptions{chapterentrydots}%
        }%
      \addtocentrydefault{chapter}{}{#2}%
      \addtocontents{toc}{\protect\endgroup}%
    }
    {\addtocentrydefault{chapter}{#1}{#2}}
}

%\usepackage{hhline}
\usepackage[labelsep=space]{caption}
%
\DeclareCaptionFormat{GOSTtable}{#2#1\\#3}
\DeclareCaptionLabelSeparator{fill}{\hfill}
\DeclareCaptionLabelFormat{fullparents}{\bothIfFirst{#1}{~}#2}
\captionsetup[table]{
     format=GOSTtable,
     labelformat=fullparents,
     labelsep=fill,
%      labelfont=it,
%      textfont=bf,
     justification=centering,
     singlelinecheck=false
     }

% центрирование колонок в таблицах
\usepackage{array}
\newcolumntype{M}[1]{>{\centering\arraybackslash}m{#1}}     

%
%\pdfmapfile{=pscyr.map}

\sloppy % переносы
% висячие строки
\usepackage[all]{nowidow}

\clubpenalty=9996
\widowpenalty=9999
\brokenpenalty=9991
\predisplaypenalty=10000
\postdisplaypenalty=1549
\displaywidowpenalty=1602


\usepackage{minted}
